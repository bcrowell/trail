\documentclass[10pt,letterpaper]{article}
\usepackage[top=0.85in,left=2.75in,footskip=0.75in]{geometry}

% amsmath and amssymb packages, useful for mathematical formulas and symbols
\usepackage{amsmath,amssymb}

% Use adjustwidth environment to exceed column width (see example table in text)
\usepackage{changepage}

% Use Unicode characters when possible
\usepackage[utf8x]{inputenc}

% textcomp package and marvosym package for additional characters
\usepackage{textcomp,marvosym}

% cite package, to clean up citations in the main text. Do not remove.
\usepackage{cite}

% Use nameref to cite supporting information files (see Supporting Information section for more info)
\usepackage{nameref,hyperref}

% line numbers
\usepackage[right]{lineno}

% ligatures disabled
\usepackage{microtype}
\DisableLigatures[f]{encoding = *, family = * }

% color can be used to apply background shading to table cells only
\usepackage[table]{xcolor}

% array package and thick rules for tables
\usepackage{array}

% create "+" rule type for thick vertical lines
\newcolumntype{+}{!{\vrule width 2pt}}

% create \thickcline for thick horizontal lines of variable length
\newlength\savedwidth
\newcommand\thickcline[1]{%
  \noalign{\global\savedwidth\arrayrulewidth\global\arrayrulewidth 2pt}%
  \cline{#1}%
  \noalign{\vskip\arrayrulewidth}%
  \noalign{\global\arrayrulewidth\savedwidth}%
}

% \thickhline command for thick horizontal lines that span the table
\newcommand\thickhline{\noalign{\global\savedwidth\arrayrulewidth\global\arrayrulewidth 2pt}%
\hline
\noalign{\global\arrayrulewidth\savedwidth}}


% Remove comment for double spacing
%\usepackage{setspace} 
%\doublespacing

% Text layout
\raggedright
\setlength{\parindent}{0.5cm}
\textwidth 5.25in 
\textheight 8.75in

% Bold the 'Figure #' in the caption and separate it from the title/caption with a period
% Captions will be left justified
\usepackage[aboveskip=1pt,labelfont=bf,labelsep=period,justification=raggedright,singlelinecheck=off]{caption}
\renewcommand{\figurename}{Fig}

% Use the PLoS provided BiBTeX style
\bibliographystyle{plos2015}

% Remove brackets from numbering in List of References
\makeatletter
\renewcommand{\@biblabel}[1]{\quad#1.}
\makeatother



% Header and Footer with logo
\usepackage{lastpage,fancyhdr,graphicx}
\usepackage{epstopdf}
%\pagestyle{myheadings}
\pagestyle{fancy}
\fancyhf{}
%\setlength{\headheight}{27.023pt}
%\lhead{\includegraphics[width=2.0in]{PLOS-submission.eps}}
\rfoot{\thepage/\pageref{LastPage}}
\renewcommand{\headrulewidth}{0pt}
\renewcommand{\footrule}{\hrule height 2pt \vspace{2mm}}
\fancyheadoffset[L]{2.25in}
\fancyfootoffset[L]{2.25in}
\lfoot{\today}

%% Include all macros below

\newcommand{\lorem}{{\bf LOREM}}
\newcommand{\ipsum}{{\bf IPSUM}}

%% END MACROS SECTION


\begin{document}
\vspace*{0.2in}

% Title must be 250 characters or less.
\begin{flushleft}
{\Large
\textbf\newline{From treadmill to trails: predicting performance of runners} 
}
\newline
% Insert author names, affiliations and corresponding author email (do not include titles, positions, or degrees).
\\
B.~Crowell\textsuperscript{1}
\\
\bigskip
\textbf{1} Natural Science Division, Fullerton College, Fullerton, CA, USA\\
\bigskip

\end{flushleft}
\section*{Abstract}


\linenumbers

% Use "Eq" instead of "Equation" for equation citations.
% For figure citations, please use "Fig" instead of "Figure".

\section*{Introduction}
Running a race on a road allows absolute measures of performance:
road runners speak of world records and personal records, and they can
easily predict whether they would be able to run a certain race safely
and with a competitive time. Trail running, however, has traditionally
been thought of as a sport in which the only valid comparison is between
different runners competing on the same course on the same day. Even the
exact measurement of distance is considered to be unimportant, since courses
and conditions vary so much. 

An extreme example is the relatively new genre
of ``vertical'' races, in which runners race up a mountain (often with about
1000 m of elevation gain). In a typical example of this type of race, the Broken
Arrow Vertical Kilometer, the competitors cover a horizontal distance of 5.0 km
while climbing a total of about 900 m. The winner in 2019 had a time of 42:46,
almost triple the time that would be expected for a state-champion high school runner in
a 5k road race. Clearly no comparison can be made here without taking into account
the amount of climbing.

In noncompetitive contexts, many runners venture onto mountain trails,
lightly dressed and with little equipment, so that it becomes important to be able
to anticipate whether they will have the endurance needed to be able to safely complete a planned route.
Again, this is impossible without some model of the effect of hill climbing.

% high school 3 mi: Joshua Schuld, PCL Dual Meets, 2020, 14:16, https://www.athletic.net/CrossCountry/Division/List.aspx?DivID=54890

This paper presents a method for predicting relative performance on trail runs --- ``relative''
meaning that we can predict the time for course A divided by the time for course B.

Traditionally, runners and hikers have described a trail using two numbers, the horizontal distance and the total elevation gain.
For example, if the route is an out-and-back voyage consisting of steady climbing to a peak and
a return, then the total elevation is simply the elevation of the peak minus the elevation of
the trailhead. If the elevation profile of the trip consisted of multiple clearly defined ascents and descents,
then one would add up the ascents. Although this two-parameter description of the route is
easy to derive from a paper topographic map, knowledge of the two numbers is not sufficient
to make a very useful estimate of the total energy expenditure.

It has been known for a long
time among the officials who measure road races that the effect
of elevation change is a nonlinear function of the grade. The following argument 
was advocated by R.~Baumel.\cite{baumel}
Consider a closed course whose elevation
profile is described by some function $y(x)$, whose derivative $y'$ is the slope of the trail $i$.
The total energy expenditure is an integrated effect of the slope, of the form $\int_0^L C(i)dx$,
where $C$ is a function that describes the energetic cost of running up or down a hill.
We will see that $C$ has been measured in laboratory experiments, but for the moment we assume
only that $C$ is a smooth function, so that for small slopes it can be well approximated by the first few terms
of its Taylor series, $C(i)\approx c_0+c_1 i+c_2 i^2$. Then for any closed loop over a distance $L$,
the contribution from the $c_1$ term vanishes, and the energy cost is $c_0 L+c_2 \int_0^L i^2 dx$.
The dependence on the slope is therefore quadratic rather than linear. For example, if we were to exaggerate the
elevation profile by a factor of 2, $y\rightarrow 2y$, then the size of the $c_2$ term would go up by
a factor of \emph{four}, not two (in the low-slope limit, on a closed course).

From conversations with runners and hikers, I have found that the result of Baumel's argument almost
always elicits total disbelief, especially when presented as a numerical example showing the extreme smallness
of the slope effect when the slope is small. One of the goals of this paper is to test this empirically.
As an alternative hypothesis, it is commonly believed that one can get a good approximation
of the relative energy cost by taking the horizontal distance and adding in a term proportional to the total elevation gain.
If the total gain is determined down to a fine enough scale (which with modern technology has become
more practical), then this hypothesis is equivalent to the assumption that the cost of running is
given by a function of the form
\begin{equation}\label{eq:gain-only}
   C_g(i) = 
  \begin{cases}
     c_0, & i\le0 \\
     c_0(1+c_g i), & i\ge 0.
  \end{cases}
\end{equation}
Popularly proposed rules are that 100 m of elevation gain is equivalent to either 400 m or 800 m of horizontal distance,
so that $c_g$ is said to be approximately in the range from 4 to 8. There is nothing mathematically impossible
about this hypothesis. A function $C(i)$ of this form evades Baumel's argument because its hockey-stick shape
is not smooth at $i=0$, and therefore cannot be approximated by its Taylor series.

Minetti \emph{et al.}\cite{minetti} have used oxygen
consumption to measure the energy expenditure of runners on a treadmill at slope $i$, for both
running and walking.
The results are expressed as $C=(1/m)d E/d s$, where $m$ is the person's body mass,
$E$ is the energy expended, and $d s$ is the increment of three-dimensional distance, which
usually differs negligibly from the increment of horizontal distance $d \ell$.
$C$ has the same units as the gravitational field.
Ref.~\cite{minetti} found significant differences in efficiency within their group of elite mountain
runners, and differences are also to be expected between this group and other athletes. This is one
of the reasons why this study presents a comparative technique, rather than an absolute method for
determining a particular runner's actual energy expenditure in units of kilocalories.

The function $C(i)$ resembles a hyperbola, with a
minimum occurring at $i\approx -0.1$ to $-0.2$. 
The asymptotes
at large positive and negative values of $i$ are interpreted in \cite{minetti} as being determined by the
efficiency of eccentric and concentric muscle contraction. 

It is convenient to describe Minetti's function $C(i)$ using a fit to the form
\begin{equation}\label{eq:minetti-fit}
  C(i) = a\left[(i^p+b)^{1/p}+\frac{i}{c}+d\right].
\end{equation}
Parameters fitted to the results of ref.~\cite{minetti} are given in table \ref{table:minetti-params}.
The purpose of using this form, rather than the polynomial fit given by \cite{minetti}, is
to make the computations degrade gracefully in cases where the limitations of GPS tracks or data from digital elevation models
produce unrealistically steep slopes. In such cases, this expression approaches the physiologically
expected asymptotic behavior. Although the present work focuses only on running, parameters for walking
are presented as well. The results for running are empirically found to be nearly independent of speed,
whereas the ones for walking are the values for the speed that was found to be most efficient for that
particular subject.

\begin{figure}[h]
\includegraphics[width=8cm]{figs/minetti/minetti.pdf}
\centering
\caption{The cost of running as a function of slope. Solid line: fit to Minetti, Eq \ref{eq:minetti-fit}. Dashed line:
Eq \ref{eq:gain-only}, with $c_0$ chosen to agree with Minetti's $C(0)$ and $c_g=6.0$.}
\label{fig:minetti}
\end{figure}

\begin{table}[h]
\caption{Parameters for equation \eqref{eq:minetti-fit}. These parameters were found by constraining
Eq \ref{eq:minetti-fit} to agree with the polynomial fits in ref.~\cite{minetti} on the following
degrees of freedom: the function is minimized at the same $i$, and has the same value of $C$ there;
the functions agree at $i=0$; and the slopes at $\pm\infty$ have the same asymptotic values.}
\begin{tabular}{lll}
   & \emph{running} & \emph{walking} \\
$a$  & 26.07 $\textup{J}/\textup{kg}\cdot\textup{m}$ & 22.91 $\textup{J}/\textup{kg}\cdot\textup{m}$ \\
$b$  & 0.03104 & 0.02621 \\
$c$ & 1.381 & 1.315 \\
$d$ & -0.06547 & -0.08317 \\
$p$ & 2.181 & 2.209
\end{tabular}
\label{table:minetti-params}
\end{table}

Nearly all real-world walking and running is
done at $-0.2 \lesssim i \lesssim 0.2$, where the graph of $C(i)$ is nearly parabolic.

\subsection*{Empirical testing}

To test these models, I use publicly available race results from the Los Angeles area.
This area has a large population and tall mountains. The large population makes it possible
to pick out a significant number of runners who have competed in several different races.
If the ratio of the runner's time on courses 1 and 2 is $t_2/t_1$, then we take this
as a measure of the ratio $E_2/E_1$ of the energy expenditure, which can be compared with
the model. It was possible to find courses with a variety of elevation profiles, allowing
a test of the dependence of the predictions on the amount of hill climbing.

\begin{table}[h]\label{table:courses}
\caption{Courses}
\begin{tabular}{lp{60mm}rrrl}
                 &       & dist.  & gain &  &  years \\
                 &       & (km)      &  (km) & CF & (20xx) \\
1 & Mount Wilson Trail Race         & 12.7 & 0.70 & 20\%        & 18-19\\
2 & Mount Baldy Run to the Top      & 10.3 & 1.21 & 47\%        & 17-18\\
3 & Broken Arrow Vertical Kilometer & 5.0  & 0.85 & 58\%        & 19 \\
4 & Pasadena Half Marathon          & 21.1 & 0.17 & 1\%         & 20 \\
5 & Agoura Hills Chesebro Half Marathon & 21.1 & 0.34 & 5\%     & 19 \\
6 & Into the Wild OC Half Marathon & 21.1 & 0.57 & 8\% & 17-18\\
7 & Griffith Park 30k               & 28.5 & 1.07 & 11\%        & 18-19 \\
8 & Revel Big Bear Half Marathon    & 21.1 & 0    & -27\% & 18-19 \\
9 & Revel Canyon City Half Marathon    & 21.1 & 0    & -4\% & 14-17
\end{tabular}
\label{table:courses}
\end{table}

Table \ref{table:courses} lists the race results used as sources of data.
Because a runner's performance can change over time due to training and aging,
the time period of the study was restricted as much as possible to January 2017 through March 2020
(before the COVID epidemic ended races other than virtual ones in California).
Digital maps projected into a horizontal plane were obtained from the race organizers' web site
or in some cases by tracing roads and trails in a Google Maps application. Elevation data
were obtained from publicly available digital elevation models (SRTM1) having a horizontal
resolution of 30 meters. (Elevation data from handheld GPS units are more difficult to
obtain from public sources, and are in any case of questionable reliability
for this purpose, since the uncertainty can be very large when all satellites are near the
horizon.) Runners' names and times were obtained by web-scraping public race results,
and runners were assumed to be the same person if their first and last names matched.
When a runner ran the same race more than once, their best time was used.

To define measures of the accuracy and precision of the model, consider a comparison
of courses 1 and 2. The observed data are the runner's times $t_1$ and $t_2$, and
the model predicts the ratio of the energy consumption $E_1/E_2$. Define
\begin{equation}
  \mathcal{E} = 100 \ln\left(\frac{t_1}{t_2}\cdot\frac{E_2}{E_1}\right).
\end{equation}
For small errors, $\mathcal{E}$ is the relative error in the prediction, expressed
as a percentage. The use of the logarithm transforms multiplicative sources of
error into additive quantities.

We pick a feature of the model that is to be tested. For example, we would like to see
whether the model does a good job of predicting the relative times for flat races
compared to steep uphill-only races. For this example, we make a list of courses
that are relatively flat (4, 5, and 6), and a list of some that are steep uphill-only
courses (2 and 3). We then find every case where the same runner did a run $i$ from the first
list and a run $j$ from the second, and compute the error $\mathcal{E}_{ij}$, which will be
positive if the runner's time in the uphill race $j$ is overpredicted by the model relative
to their time in the flat race $i$. From the resulting list of $\mathcal{E}$ values for
this sample, we find the median $\widetilde{\mathcal{E}}$ and the mean absolute value $\langle|\mathcal{E}|\rangle$.

\subsection*{Model of endurance}

Animals run more slowly at long distances, and mathematical modeling of this fact dates back
about a century.\cite{hill} Recent workers have described methods for fitting parameters
to the data for individual runners,\cite{rapoport}\cite{emig} which for example allows
a first-time marathon runner to estimate an appropriate pace. In the present work, there is
not enough data available to allow this kind of individualized description of the runners.
For this reason, I have concentrated on data from a narrow range of middle distances,
with the total energy expenditure being close to that of a flat half-marathon road race.
But the endurance required for these races does vary, and this makes it desirable to have
some rough method of compensating for the variation in pace with distance. Here I describe
a very simple model that has the following characteristics that make it suitable for this
study: (1) its parameters are universal rather than fits to the characteristics of an individual;
(2) its output is purely multiplicative, i.e., it gives a correction factor $\kappa$ to the predicted speed.
The model is essentially a variation on the one by Rapoport,\cite{rapoport} with some modifications
to suit these purposes.

First we compute an equivalent distance $d$, which is the distance of flat running that would
require the same energy expenditure as the actual run. If the runner's time is $t$, then
$v=d/t$ has dimensions of speed, but is in fact a measure of energy per unit time, or power.
We then have
\begin{equation}\label{eq:fuel-model-1}
  Pt=d/\epsilon,
\end{equation}
where $P$ is the power and $\epsilon$ is a measure of the runner's efficiency. For example,
a recreational runner with a slight roll of belly fat will have a lower value of $\epsilon$ because of
the increased energetic cost of transporting the additional body weight. Although it would seem that
we are now introducing an individualized parameter $\epsilon$, the model is designed so that at the
end of the calculation, cancellations occur that allow $\kappa$ to be predicted on a universal basis.

The power $P$ depends on aerobic fitness and on the proportions of fat and carbohydrates being
burned in aerobic metabolism. Fat burning is slower than carbohydrate burning by a factor $\beta\approx 0.4$.\cite{rapoport}
If we let $f$ be the fraction of energy production from carbohydrates, then
\begin{equation}\label{eq:fuel-model-2}
  P = A[f+\beta(1-f)],
\end{equation}
where the proportionality constant $A$ is another per-individual parameter that it will be
possible to normalize away later. This expression's linearity in $f$ is an approximation to
results from real-world data that provide evidence for slightly nonlinear behavior.\cite{rapoport}

The runner's supply of carbohydrates $c$ is limited by the amount of glycogen that can be stored
in the liver and the leg muscles. If $f$ is chosen optimally, then there will be some distance
$d_c=c\epsilon$ that can be run with pure carbohydrate fuel, while longer distances will require $f<1$.
Thus,
\begin{equation}\label{eq:fuel-model-3}
  f = \begin{cases}
    1, & d\le d_c \\
    c/At, & d\ge d_c.
\end{cases}
\end{equation}
Under these assumptions, the runner's speed will be the same in races at all distances less than $d_c$,
which is unrealistic. We will first work out the consequences of Eq \ref{eq:fuel-model-1}-\ref{eq:fuel-model-3}
and the introduce a simple elaboration that more realistically reproduces the effects of fatigue.

Solving Eq \ref{eq:fuel-model-1}-\ref{eq:fuel-model-3} and expressing $\kappa$ as a correction factor
relative to the short-distance maximum speed $v_m=A\epsilon$, we find
\begin{equation}\label{eq:kappa-simple}
  \kappa_0 = \begin{cases}
    1, & d\le d_c \\
    \frac{\beta}{1-(1-\beta)d_c/d}, & d\ge d_c.
  \end{cases}
\end{equation}
This depends on the universal parameter $\beta\approx0.4$ and also on the critical distance $d_c$.
The latter is a measure of endurance and does depend on individual factors such as
body composition and training, as well as on strategies such as carbohydrate loading.
However, for the sample of recreational athletes studied here, I hypothesize that one can fix a universal value of $d_c$ lying somewhere
around the half-marathon distance, and find a reasonable description of
real-world data.

It is not true in reality that runners can maintain the same pace at any of the distances below
$d_c$, for which glycogen suffices. As the distance increases from 5 km to the half-marathon
distace of 21 km, one observes a decrease in speed which, as originally observed by Hill,\cite{hill} appears
linear on a graph of speed versus the logarithm of distance. In the men's and women's world-record
times, this decrease is about 5\%. The graph then shows a knee, like the one described by Eq \ref{eq:kappa-simple}.
The more gradual decrease for distances before the knee is generically described as being due to fatigue, which is
a complicated and poorly understood phenomenon involving a variety of factors, many of which are mediated
by the central nervous system rather than by any change at the chemical or tissue level. As an \emph{ad hoc}
correction, we multiply the result of Eq \ref{eq:kappa-simple} by a factor
controlled by a small parameter $Q$:
\begin{equation}\label{eq:kappa}
  \kappa = \left(\frac{1+(1-3Q)d/d_c}{1+d/d_c}\right) \kappa_0.
\end{equation}
The factor of 3 is introuced so that $Q$ is approximately equal to the reduction in speed between a 5k and a half-marathon,
and we set $Q=0.05$.

Empirically, for the mostly recreational runners studied here, a reasonable description of the data is achieved when
$d_c$ is set to the half-marathon distance, which is the value adopted in this work.
Fig \ref{fig:my-times} shows the quality of the fit for some real-world data.
A reduction of $d_c$ by about 24\% gave somewhat better agreement (zero median error) with the data in the sample of runners studied
here,  when times in a hilly 30k trail race were
predicted based on times in shorter races in which the energy expenditure was roughly equivalent to that of
a flat half marathon.
On the other hand, much higher values of $d_c$ might be more appropriate for higher-level endurance runners.
For example, Eliud Kipchoge's personal-record
speed is only 8\% lower in the marathon than in the 1500 m. This can only be reproduced in this model if $d_c$
is roughly marathon distance for him.

\begin{figure}[h]
\includegraphics[width=8cm]{figs/my-times/my-times.pdf}
\centering
\caption{The author's personal-best speeds $v/v_m$ on various courses, versus equivalent distance $d$.
The curve is the function defined by Eq \ref{eq:kappa}, with $d_c$ set to a half-marathon distance.
The times are from a mixture of road and trail courses, and a mixture of races and running against the clock.
The equivalent distances were determined from the horizontal distances using the curvlinear
function $C(i)$ in Eq \ref{eq:minetti-fit}, which is based on treadmill data.}
\label{fig:my-times}
\end{figure}

Although fitting parameters to individual runners' characteristics is not the main purpose of this work, doing
so is very easy with this model, due to its purely multiplicative structure.
When data are viewed in the format used in Fig \ref{fig:my-times}, as a log-log plot of speed versus distance,
the standard curve $\kappa(d)$ is simply slid around horizontally and vertically to match the runner's $v_m$ and $d_c$.


\bibliography{trail}


\end{document}

