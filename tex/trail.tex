\documentclass[10pt,letterpaper]{article}
\usepackage[top=0.85in,left=2.75in,footskip=0.75in]{geometry}

% amsmath and amssymb packages, useful for mathematical formulas and symbols
\usepackage{amsmath,amssymb}

% Use adjustwidth environment to exceed column width (see example table in text)
\usepackage{changepage}

% Use Unicode characters when possible
\usepackage[utf8x]{inputenc}

% textcomp package and marvosym package for additional characters
\usepackage{textcomp,marvosym}

% cite package, to clean up citations in the main text. Do not remove.
\usepackage{cite}

% Use nameref to cite supporting information files (see Supporting Information section for more info)
\usepackage{nameref,hyperref}

% line numbers
\usepackage[right]{lineno}

% ligatures disabled
\usepackage{microtype}
\DisableLigatures[f]{encoding = *, family = * }

% color can be used to apply background shading to table cells only
\usepackage[table]{xcolor}

% array package and thick rules for tables
\usepackage{array}

% create "+" rule type for thick vertical lines
\newcolumntype{+}{!{\vrule width 2pt}}

% create \thickcline for thick horizontal lines of variable length
\newlength\savedwidth
\newcommand\thickcline[1]{%
  \noalign{\global\savedwidth\arrayrulewidth\global\arrayrulewidth 2pt}%
  \cline{#1}%
  \noalign{\vskip\arrayrulewidth}%
  \noalign{\global\arrayrulewidth\savedwidth}%
}

% \thickhline command for thick horizontal lines that span the table
\newcommand\thickhline{\noalign{\global\savedwidth\arrayrulewidth\global\arrayrulewidth 2pt}%
\hline
\noalign{\global\arrayrulewidth\savedwidth}}


% Remove comment for double spacing
%\usepackage{setspace} 
%\doublespacing

% Text layout
\raggedright
\setlength{\parindent}{0.5cm}
\textwidth 5.25in 
\textheight 8.75in

% Bold the 'Figure #' in the caption and separate it from the title/caption with a period
% Captions will be left justified
\usepackage[aboveskip=1pt,labelfont=bf,labelsep=period,justification=raggedright,singlelinecheck=off]{caption}
\renewcommand{\figurename}{Fig}

% Use the PLoS provided BiBTeX style
\bibliographystyle{plos2015}

% Remove brackets from numbering in List of References
\makeatletter
\renewcommand{\@biblabel}[1]{\quad#1.}
\makeatother



% Header and Footer with logo
\usepackage{lastpage,fancyhdr,graphicx}
\usepackage{epstopdf}
%\pagestyle{myheadings}
\pagestyle{fancy}
\fancyhf{}
%\setlength{\headheight}{27.023pt}
%\lhead{\includegraphics[width=2.0in]{PLOS-submission.eps}}
\rfoot{\thepage/\pageref{LastPage}}
\renewcommand{\headrulewidth}{0pt}
\renewcommand{\footrule}{\hrule height 2pt \vspace{2mm}}
\fancyheadoffset[L]{2.25in}
\fancyfootoffset[L]{2.25in}
\lfoot{\today}

%% Include all macros below

\newcommand{\lorem}{{\bf LOREM}}
\newcommand{\ipsum}{{\bf IPSUM}}

%% END MACROS SECTION


\begin{document}
\vspace*{0.2in}

% Title must be 250 characters or less.
\begin{flushleft}
{\Large
\textbf\newline{From treadmill to trails: predicting performance of runners} 
}
\newline
% Insert author names, affiliations and corresponding author email (do not include titles, positions, or degrees).
\\
B.~Crowell\textsuperscript{1}
\\
\bigskip
\textbf{1} Natural Science Division, Fullerton College, Fullerton, CA, USA\\
\bigskip

\end{flushleft}
\section*{Abstract}


\linenumbers

% Use "Eq" instead of "Equation" for equation citations.
% For figure citations, please use "Fig" instead of "Figure".

\section*{Introduction}
Running a race on a road allows absolute measures of performance:
road runners speak of world records and personal records, and they can
easily predict whether they would be able to run a certain race safely
and with a competitive time. Trail running, however, has traditionally
been thought of as a sport in which the only valid comparison is between
different runners competing on the same course on the same day. Even the
exact measurement of distance is considered to be unimportant, since courses
and conditions vary so much. 

An extreme example is the relatively new genre
of ``vertical'' races, in which runners race up a mountain (often with about
1000 m of elevation gain). In a typical example of this type of race, the Broken
Arrow Vertical Kilometer, the competitors cover a horizontal distance of 5.0 km
while climbing a total of about 900 m. The winner in 2019 had a time of 42:46,
almost triple the time that would be expected for a state-champion high school runner in
a 5k road race. In noncompetitive contexts, many runners venture onto mountain trails,
lightly dressed and with little equipment, so that it becomes important to be able
to anticipate whether they will have the endurance needed to be able to safely complete a planned route.

% high school 3 mi: Joshua Schuld, PCL Dual Meets, 2020, 14:16, https://www.athletic.net/CrossCountry/Division/List.aspx?DivID=54890

This paper presents a method for predicting relative performance on trail runs --- ``relative''
meaning that if we know the runner's time for course A, we can predict their time on course B.
Traditionally, runners and hikers would compare the lengths

\subsection*{Model of endurance}

Animals run more slowly at long distances, and mathematical modeling of this fact dates back
about a century.\cite{hill} Recent workers have described methods for fitting parameters
to the data for individual runners,\cite{rapoport}\cite{emig} which for example allows
a first-time marathon runner to estimate an appropriate pace. In the present work, there is
not enough data available to allow this kind of individualized description of the runners.
For this reason, I have concentrated on data from a narrow range of middle distances,
with the total energy expenditure being close to that of a flat half-marathon road race.
But the endurance required for these races does vary, and this makes it desirable to have
some rough method of compensating for the variation in pace with distance. Here I describe
a very simple model that has the following characteristics that make it suitable for this
study: (1) its parameters are universal rather than fits to the characteristics of an individual;
(2) its output is purely multiplicative, i.e., it gives a correction factor $\kappa$ to the predicted speed.
The model is essentially a variation on the one by Rapoport,\cite{rapoport} with some modifications
to suit these purposes.

First we compute an equivalent distance $d$, which is the distance of flat running that would
require the same energy expenditure as the actual run. If the runner's time is $t$, then
$v=d/t$ has dimensions of speed, but is in fact a measure of energy per unit time, or power.
We then have
\begin{equation}\label{eq:fuel-model-1}
  Pt=d/\epsilon,
\end{equation}
where $P$ is the power and $\epsilon$ is a measure of the runner's efficiency. For example,
a recreational runner with a slight roll of belly fat will have a lower value of $\epsilon$ because of
the increased energetic cost of transporting the additional body weight. Although it would seem that
we are now introducing an individualized parameter $\epsilon$, the model is designed so that at the
end of the calculation, cancellations occur that allow $\kappa$ to be predicted on a universal basis.

The power $P$ depends on aerobic fitness and on the proportions of fat and carbohydrates being
burned in aerobic metabolism. Fat burning is slower than carbohydrate burning by a factor $\beta\approx 0.4$.\cite{rapoport}
If we let $f$ be the fraction of energy production from carbohydrates, then
\begin{equation}\label{eq:fuel-model-2}
  P = A[f+\beta(1-f)],
\end{equation}
where the proportionality constant $A$ is another per-individual parameter that it will be
possible to normalize away later. This expression's linearity in $f$ is an approximation to
results from real-world data that provide evidence for slightly nonlinear behavior.\cite{rapoport}

The runner's supply of carbohydrates $c$ is limited by the amount of glycogen that can be stored
in the liver and the leg muscles. If $f$ is chosen optimally, then there will be some distance
$d_c=c\epsilon$ that can be run with pure carbohydrate fuel, while longer distances will require $f<1$.
Thus,
\begin{equation}\label{eq:fuel-model-3}
  f = \begin{cases}
    1, & d\le d_c \\
    c/At, & d\ge d_c.
\end{cases}
\end{equation}
Under these assumptions, the runner's speed will be the same in races at all distances less than $d_c$,
which is unrealistic. We will first work out the consequences of Eq \ref{eq:fuel-model-1}-\ref{eq:fuel-model-3}
and the introduce a simple elaboration that more realistically reproduces the effects of fatigue.

Solving Eq \ref{eq:fuel-model-1}-\ref{eq:fuel-model-3} and expressing $\kappa$ as a correction factor
relative to the short-distance maximum speed $v_m=A\epsilon$, we find
\begin{equation}\label{eq:kappa-simple}
  \kappa_0 = \begin{cases}
    1, & d\le d_c \\
    \frac{\beta}{1-(1-\beta)d_c/d}, & d\ge d_c.
  \end{cases}
\end{equation}
This depends on the universal parameter $\beta\approx0.4$ and also on the critical distance $d_c$.
The latter is a measure of endurance and does depend on individual factors such as
body composition and training, as well as on strategies such as carbohydrate loading.
However, for the sample of recreational athletes studied here, I hypothesize that one can fix a universal value of $d_c$ lying somewhere
around the half-marathon distance, and find a reasonable description of
real-world data.

It is not true in real-world conditions that runners can maintain the same pace at any distance below
$d_c$, for which glycogen suffices. As the distance increases from 5 km to the half-marathon
distace of 21 km, one observes a decrease in speed which, as originally observed by Hill,\cite{hill} appears
linear on a graph of speed versus the logarithm of distance. In the men's and women's world-record
times, this decrease is about 5\%. The graph then shows a knee, like the one described by Eq \ref{eq:kappa-simple}.
The decrease for distances before the knee is generically described as being due to fatigue, which is
a complicated and poorly understood phenomenon involving a variety of factors, many of which are mediated
by the central nervous system rather than by any change at the chemical or tissue level. A reasonable
evolutionary hypothesis is that the CNS effectively uses exertion up to a certain time to extrapolate
the need for further exertion, husbanding energy resources and preventing damage in a prediction-feedback
loop that is not under voluntary control. With this motivation, we introduce an elaboration 
in which the result of Eq \ref{eq:kappa-simple} is replaced by its average over all distances from
$d$ up to $2d$,
\begin{equation}\label{eq:kappa}
  \kappa = \frac{1}{d}\int_d^{2d} \kappa_0(d') dd'.
\end{equation}
This can be conveniently implemented in software using the indefinite integral $\int dx/(1-1/x)=\ln(x-1)+x+c$.

Empirically, for the mostly recreational runners studied here, a reasonable description of the data is achieved when
$d_c$ is set to the half-marathon distance, which is the value adopted in this work.
Fig \ref{fig:my-times} shows the quality of the fit for some real-world data.
A reduction of $d_c$ by about 24\% gave somewhat better agreement (zero median error) with the data in the sample of runners studied
here,  when times in a hilly 30k trail race were
predicted based on times in shorter races in which the energy expenditure was roughly equivalent to that of
a flat half marathon.
On the other hand, much higher values of $d_c$ might be more appropriate for higher-level endurance runners.
Indeed, the model fails badly for some elite athletes. For example, Eliud Kipchoge's personal-best
speed is only 8\% lower in the marathon than in the 1500 m. This cannot be reproduced with any physiologically
reasonable value of $d_c$, possibly indicating that the fatigue model in Eq \ref{eq:kappa} is invalid for Kipchoge.

\begin{figure}[h]
\includegraphics[width=8cm]{figs/my-times/my-times.pdf}
\centering
\caption{The author's personal-best speeds $v/v_m$ on various courses, versus equivalent distance $d$.
The curve is the function defined by Eq \ref{eq:kappa}, with $d_c$ set to a half-marathon distance.
The times are from a mixture of road and trail courses, races and running against the clock.}
\label{fig:my-times}
\end{figure}

Although fitting parameters to individual runners' characteristics is not the main purpose of this work, doing
so is very easy with this model, due to its purely multiplicative structure.
When data are viewed in the format used in Fig \ref{fig:my-times}, as a log-log plot of speed versus distance,
the standard curve $\kappa(d)$ is simply slid around horizontally and vertically to match the runner's $v_m$ and $d_c$.

\bibliography{trail}


\end{document}

